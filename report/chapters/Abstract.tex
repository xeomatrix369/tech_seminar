\chapter*{Abstract}
\addcontentsline{toc}{chapter}{Abstract}
\begin{tikzpicture}[remember picture, overlay]
	% Draw border
	\draw[line width=0.5mm, color=black!20] (current page.south west) rectangle (current page.north east);
	% Add watermark text
	\node[rotate=45, scale=8, color=black!10, opacity=0.2] at (current page.center) {Mohammed Hamza};
\end{tikzpicture}

\addborder
\begin{center}
	\vspace{5mm}
	\MakeUppercase{\textbf{DATA CONVERTERS IN VLSI SYSTEMS}}\\
	\vspace{5mm}
	\memberA \\
	\vspace{5mm}
	Guide: \guideA \\
	\vspace{5mm}
\end{center}

\noindent \textbf{Keywords:}\\
Data Converters, VLSI Systems, Analog-to-Digital Converter (ADC), Digital-to-Analog Converter (DAC), Successive Approximation Register (SAR), Sigma-Delta ADC, Signal Processing, Electronic Systems.
\\

In this report, we endeavor to understand Data Converters in VLSI Systems, 
focusing on the theoretical aspects of the Analog-to-Digital Converter (ADC) and Digital-to-Analog Converter (DAC).
We explore the principles of operation, design considerations, and performance metrics of these converters.
The report also delves into the different architecture of ADCs and DACs,
including their working principles, advantages, and disadvantages.
We discuss the various types of ADCs, such as Successive Approximation Register (SAR)
and Sigma-Delta ADCs, and the different types of DACs, including R-2R Ladder DACs and Current Steering DACs.
The report provides a comprehensive overview of the design and implementation of data converters in VLSI systems.The report highlights the importance of data converters in bridging the gap between analog and digital domains.enabling efficient signal processing and communication in electronic systems.The report is structured to provide transistor level implementation of Circuits in DACs and ADCs.
