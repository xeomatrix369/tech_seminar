\chapter{Literature Review}
\addborderb
Data converters, primarily Analog-to-Digital Converters (ADCs) and Digital-to-Analog Converters (DACs), are essential components in modern electronic systems, enabling seamless interfacing between the analog real world and digital processing units. Over the past decades, significant advancements have been made in converter architectures, performance, and integration.

Early data converters were based on simple resistor ladder and integrating architectures, such as the successive approximation register (SAR) ADC and the binary-weighted DAC. These architectures offered moderate resolution and speed, suitable for early digital systems. As technology progressed, pipeline and sigma-delta (\(\Sigma\Delta\)) architectures emerged, providing higher resolution and faster conversion rates. Pipeline ADCs are widely used in applications requiring high speed and moderate resolution, such as communication systems, while sigma-delta converters excel in high-resolution, low-frequency applications like audio and instrumentation.

Recent literature has focused on improving the power efficiency, linearity, and area of data converters. Techniques such as calibration, digital correction, and time-interleaving have been proposed to overcome non-idealities and enhance performance. The scaling of CMOS technology has enabled the integration of high-performance data converters into system-on-chip (SoC) solutions, further driving research into low-voltage and low-power designs.

Emerging trends include the use of machine learning for calibration, the development of asynchronous and event-driven converters, and the exploration of new materials and device structures to push the boundaries of speed and resolution. The literature indicates a continuous trade-off between speed, resolution, power consumption, and area, with ongoing research aimed at optimizing these parameters for various application domains.